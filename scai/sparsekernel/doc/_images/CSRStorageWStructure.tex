\documentclass{article}
\usepackage{amsfonts}              % fuer zusätzliche math. Symbole
\usepackage{amssymb}               % weitere math. Symbole
\usepackage{amsmath}
\usepackage{mathtools}
\usepackage{tikz}
\usepackage{verbatim}
\usepackage[active,tightpage]{preview}

\usetikzlibrary{decorations.pathreplacing}
\usetikzlibrary{arrows,shapes,positioning,shadows,trees,matrix}
\PreviewEnvironment{tikzpicture}
\setlength\PreviewBorder{10pt}

\usetikzlibrary{chains}

\begin{document}

\begin{tikzpicture}[
	node distance=1mm,
	mymathmatrix/.style={
		matrix of math nodes,
		inner sep=0pt,
		nodes={text width=2em,align=center,font=\mathstrut},
		left delimiter=(,right delimiter=)
	},
	every node/.style={font=\sffamily}
]

% Values
\matrix (values) [mymathmatrix]{
6.0&  4.0&  0.0&  7.0&  9.0&  4.0&  3.0&  2.0&  5.0&  2.0&  1.0& 1.0&  2.0\\
};

\matrix (ja) [mymathmatrix, below=1em of values]{
0&    3&    1&    0&    2&    3&    3&    0&    1&    0&          3&    1& 3 \\
};

\matrix (ia) [mymathmatrix, below=1em of ja]{
0&    ~&     2&   ~&     4&    ~&   6&     ~&   ~&    9&     11&   11&    13 \\
};

\node(eq_values) [left=1em of values, anchor=east]{=};
\node(eq_ja) [left=1em of ja, anchor=east]{=};
\node(eq_ia) [left=1em of ia, anchor=east]{=};
\node(eq_col) [above=1em of eq_values]{=};
\node(eq_row) [above=1em of eq_col]{=};
\node(eq_val) [above=1em of eq_row]{=};

\node [left=6em of eq_val, anchor=west]{numValues};
\node [left=6em of eq_row, anchor=west]{numRows};
\node [left=6em of eq_col, anchor=west]{numColumns};
\node [left=6em of eq_values, anchor=west]{values};
\node [left=6em of eq_ja, anchor=west]{ja};
\node [left=6em of eq_ia, anchor=west]{ia};

\node [right=1em of eq_col]{4};
\node [right=1em of eq_row]{7};
\node [right=1em of eq_val]{13};

\end{tikzpicture}

\end{document}

