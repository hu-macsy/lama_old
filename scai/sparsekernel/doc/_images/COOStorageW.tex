\documentclass{article}
\usepackage{amsfonts}              % fuer zusätzliche math. Symbole
\usepackage{amssymb}               % weitere math. Symbole
\usepackage{amsmath}
\usepackage{mathtools}
\usepackage{tikz}
\usepackage{verbatim}
\usepackage[active,tightpage]{preview}

\usetikzlibrary{decorations.pathreplacing}
\usetikzlibrary{arrows,shapes,positioning,shadows,trees,matrix}
\PreviewEnvironment{tikzpicture}
\setlength\PreviewBorder{10pt}

\usetikzlibrary{chains}

\newcommand\f[1]{\textbf{#1}}

\begin{document}

\begin{tikzpicture}[
	node distance=1mm,
	mymathmatrix/.style={
		matrix of math nodes,
		inner sep=0pt,
		nodes={text width=2em,align=center,font=\mathstrut},
		left delimiter=(,right delimiter=)
	},
	every node/.style={font=\sffamily}
]

% Linke Spalte
\matrix (A1) [mymathmatrix]{
\f{6.0} & \f{0.0} & \f{9.0} &  \f{3.0} \\  
*       & *       & *       &  4.0     \\
7.0     & *       & *       &  *       \\
*       & *       & *       &  4.0     \\
2.0     & 5.0     & *       &  *       \\
2.0     & *       & *       &  1.0     \\
*       & *       & *       &  *       \\
*       & 1.0     & *       &  2.0     \\
};


\end{tikzpicture}

\end{document}

